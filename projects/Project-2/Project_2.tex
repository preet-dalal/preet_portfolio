\documentclass[11pt,a4paper]{article}
\usepackage{amsmath,amssymb,amsfonts}
\usepackage{graphicx}
\usepackage{hyperref}
\usepackage{geometry}
\usepackage{float}
\usepackage{booktabs}
\usepackage{caption}
\usepackage{subcaption}
\usepackage{multirow}
\usepackage{bm}
\geometry{margin=1in}

\title{\textbf{Project 2: MCMC Fitting of the $\Lambda$CDM Model using Observational Data}}
\author{Preet Dalal}
\date{\today}


\begin{document}
\maketitle

\section*{Overview}

This project demonstrates a complete Bayesian inference pipeline for the standard $\Lambda$CDM cosmological model using a combination of Cosmic Chronometers (CC), Baryon Acoustic Oscillations (BAO), and the UNION3 Type Ia supernova dataset \footnote{The same method can be applied to other Type IA Supernovae datasets like Pantheon+SH0ES and DES5Y.}.
The objective is to re--produce standard literature by statistically constraining cosmological parameters by performing a full MCMC analysis with convergence diagnostics, goodness-of-fit tests, and posterior characterization.

This pipeline forms the numerical backbone of my cosmological data analysis and has been further extended to interacting dark energy models in modified $f(Q)$ gravity whose manuscript has been submitted to peer-reviewd journals for publications.

\section*{$\Lambda$CDM Model}

If we ignore the radiation density parameter $\Omega_{r0} \approx 8.4 \times 10^{-4}$ then our Hubble parameter is given by
\[
H(z) = H_0 \sqrt{\Omega_{m0}(1+z)^3 +  1 - \Omega_{m0}},
\]
where $\Omega_{m0}$ is the matter density parameter and $z$ is the redshift. 


\section{Datasets}

\subsection{Cosmic Chronometer (CC)}

I used the latest 32 CC measurements of $H(z)$ spanning from $0.070<z< 1.965$. Out of the 32 points, 15 are correlated and the remaining 17 are uncorrelated. The $\chi^2$ statistic is given by 
\[
\chi^2_{\text{CC}} = 
\sum_{i=1}^{N_u}
\frac{\left[ H_{\text{th}}(z_i)-H_{\text{obs}}(z_i) \right]^2}{\sigma_i^2}
+
\Delta \bm{H}^{T}\mathbf{C}_{\text{CC}}^{-1}\Delta \bm{H},
\]
where the first term corresponds to the uncorrelated points and the second term uses the inverse covariance matrix for the correlated subset. 


\subsection{Baryonic Acoustic Oscillations (BAO)}

In this project, I use the latest DESI DR2 BAO data that measures three ratios : $ D_M/r_s $, $ D_H/r_s $ and $ D_V/r_s $. Here, $ r_s $ is the sound horizon radius at the drag epoch $ z_d $ which depends on the speed of sound $ c_s(z) $ in the pervading fluid. 
\[
 r_s=\int_{z_d}^\infty\frac{c_s(z)}{H(z)} dz\;.
\]
   
The transverse comoving distance, the Hubble distance and the volume-averaged distance are defined as
\begin{gather*}
    D_M(z) = c\int_0^z \frac{dz'}{H(z')}\;,\\
    D_H(z) = \frac{c}{H(z)}\;,\\
    D_V(z) = \left( zD_H(z)D_M^2(z) \right)^{1/3}\;.
\end{gather*}

The $\chi^2$ statistic for the BAO data is given as
\[
\chi^2_{\text{BAO}} =
\Delta \bm{D}^{T}
\mathbf{C}_{\text{BAO}}^{-1}
\Delta \bm{D},
\]
where $\Delta D$ is the residual vector and $C^{-1}_{\rm BAO}$ is the inverse of covariance matrix. 

\subsection{Type Ia Supernovae}

Type Ia supernovae (SNeIa) serve as standard candles, allowing the deduction of luminosity distance $d_L(z) = (1+z)D_M(z)$ through the observed distance modulus $ \mu = m - M $, where $ m \text{ and } M $ are the apparent magnitude and the absolute magnitude. The theoretical distance modulus is given by given by
\begin{equation*}
    \mu(z) = 5\log_{10}\left( \frac{d_L(z)}{1\text{ Mpc}} \right)+25\;.
\end{equation*}
I have used the UNION3 SNe Ia dataset to constrain the parameter space. Union 3 is a dataset consisting of 22 points, which is a compressed version of the compilation of 2087. The redshift range for this dataset is $ 0.01 < z < 2.3 $.


The $\chi^2$ statistic for the UNION3 data is given as
\[
\chi^2_{\text{SN}} =
\Delta \bm{\mu}^{T}
\mathbf{C}_{\text{SN}}^{-1}
\Delta \bm{\mu},
\]
where $\Delta \mu$ is the residual vector and $C^{-1}_{\rm SN}$ is the inverse of covariance matrix. 

\section*{Likelihood Construction}


I perform Bayesian parameter estimation using Markov Chain Monte Carlo (MCMC)  methods implemented in the \texttt{emcee} sampler. The posterior distribution of the model parameters denoted collectively by $ \theta $ is sampled according to the Bayes theorem
\begin{equation*}
    \mathcal{P}(\theta | \text{data}) \propto \mathcal{L}(\text{data} | \theta)\,\pi(\theta)\;,
\end{equation*}
where $ \mathcal{L} $ is the likelihood and $ \pi(\theta) $ the prior. The likelihood is constructed from the $\chi^{2}$ statistic,
\begin{equation*}
    \mathcal{L}(\text{data} \mid \theta) \propto \exp\left(-\frac{1}{2}\chi^{2}(\theta)\right)\;,
\end{equation*}

where the total chi-square is 
\[
\chi^2 =
\chi^2_{\text{CC}} +
\chi^2_{\text{BAO}} +
\chi^2_{\text{SN}}
.
\]
We adopt Gaussian priors from the observational constraints:
\begin{gather*}
    H_{0} \sim \mathcal{N}(70,\,5^{2}) \quad;\quad 50.0 < H_{0} < 90.0 \\
    \Omega_{m0} \sim \mathcal{N}(0.3,\,0.05^{2}) \quad;\quad 0.0 < \Omega_{m0} < 0.5\\
\end{gather*}

\section*{MCMC Pipeline}


- Definition of parameter vector $\bm{\theta} = (H_0, \Omega_{m0})$. \\
- Construction of CC, BAO, and UNION3 likelihoods. \\
Implementation of $\ln \mathcal{L}$, $\ln \pi$, and $\ln \mathcal{P}$. \\
- Sampling using \texttt{emcee}. \\
- Burn-in removal and chain thinning.
- Convergence diagnostics: Gelman--Rubin $\hat{R}$, autocorrelation time. \\
- Posterior visualization and statistics.



\section*{Trace Plots}

\begin{figure}[H]
\centering
\includegraphics[width=0.9\textwidth]{Project-2/LCDM_UNION3_trace.jpeg}
\caption{Trace plots showing the evolution of MCMC chains for $\Lambda$CDM parameters, demonstrating convergence and good mixing.}
\end{figure}


\section*{Autocorrelation Time and Effective Sample Size}

\begin{figure}[H]
\centering
\begin{subfigure}{0.48\textwidth}
  \includegraphics[width=\textwidth]{Project-2/running_tau.jpeg}
  \caption{Autocorrelation time.}
\end{subfigure}
\hfill
\begin{subfigure}{0.48\textwidth}
  \includegraphics[width=\textwidth]{Project-2/running_ess.jpeg}
  \caption{Effective sample size.}
\end{subfigure}
\caption{Autocorrelation and effective sample size diagnostics for the MCMC chains.}
\end{figure}



\section*{Corner Plot}

\begin{figure}[H]
\centering
\includegraphics[width=0.48\textwidth]{Project-2/LCDM_UNION3_Corner.jpeg}
\caption{Corner plot showing posterior distributions and parameter correlations for $\Lambda$CDM using CC + BAO + UNION3 data. The $\chi^2_{\rm min}$ of the run is found to be 68.22 and hence the reduced $\chi^2_{ \rm min}$ of the fit is 1.18. 
}
\end{figure}


\section*{Correlation Matrix}



\begin{figure}[H]
\centering
\includegraphics[width=0.6\textwidth]{Project-2/LCDM_UNION3_correlation_matrix.jpeg}
\caption{Correlation matrix of the $\Lambda$CDM parameters derived from posterior samples.}
\end{figure}




\section*{Summary}


- Full Bayesian inference pipeline implemented. \\
- Multiple datasets combined consistently. \\
- Convergence, autocorrelation, ESS, and $\hat{R}$ verified. \\
- Goodness-of-fit quantified via $\chi^2$. \\
- Correlations and degeneracies explicitly analyzed. \\
- Framework extended to interacting dark energy in $f(Q)$ gravity. 

\vspace{0.3cm}

\end{document}
